\section{Recommender Systems}

\subsection{Brief History}

    The idea of using a computer-based filtering system according to users' preferences dates back to 
    the 1960s, when the early techniques were published by the name of ``Selective Dissemination of Information", 
    where user-prompted keywords where employed to sort and filter documents \cite{1963SDI}. These techniques 
    based on document representations created the foundations of what came to be known as 
    content-based recommendations \cite{2016BeyondMatrixCompletion}. 

    Later in the 1980s \textit{Xerox PARC} implemented the \textit{Tapestry} system which was designed to 
    filter emails according to user-specified rules. Such system even brought the idea of users 
    classifying other readers' messages, thus introducing the concept of collaborative filtering
    \cite{2016BeyondMatrixCompletion}.

    By the 2000s the business impact of such systems started drawing attention after 
    \textit{Amazon.com} launched its online retailer store, whose revenue heavily 
    relied on its recommendation engine. E-commerce companies then saw 
    the strategic value of RS in such a way that, in 2007, \textit{Netflix} launched an open challenge 
    rewarding 1 million dollars to the first team who managed to improve its recommendation engine in 
    $10 \%$. More than 5000 teams registered for the competition and only in 2009 was the goal achieved 
    by a group of researches, being a turning point in RS research \cite{2007TheNetflixPrize}.
    
    Despite bringing academic research forward, one of the consequences of the challenge was to steer 
    the focus of research to the task of estimating users' preferences of unseen items or to provide users with 
    ranked item lists. However, in practice RS can serve a variety of purposes other than these tasks and, 
    therefore, a broader overview of the recommendation problem is needed in order to understand its 
    true value.

  \subsection{The Recommendation Problem}

    As defined in \cite{2016BeyondMatrixCompletion}, the recommendation problem can be defined as: 

    \begin{quote}
      \textit{Find a sequence of conversational actions and item recommendations for each particular user that 
      optimizes the overall goal over the specified timeframe.}  
    \end{quote}
    
    Consequently, this problem is formed by three components: an overall goal, a set of actions and 
    an optimization timeframe. 

    The \textbf{overall goal} represents a set of measures to be optimized by the RS. Traditionally, they are designed 
    as accuracy or ranking metrics, such as precision and recall, but key performance indicators (KPI) have 
    also been used to define RS optimization goals that are more oriented towards business values \cite{2019BusinessValue}.

    % As we can see, the business value of RS has grown over the years as more companies apply personalizations
    % as a strategy to improve their relationship with customers. Although is not always feasible to 
    % measure the true value of a RS, some business value measurements according to  
    % are presented in Figure \ref{fig:business_value}.

    % \begin{figure}
    %   \includegraphics[width=\linewidth]{./../img/rs_business_value.png}
    %   \caption{Overview of RS business value measurement approaches \cite{2019BusinessValue}.}
    %     \label{fig:business_value}
    % \end{figure}
    
    \textit{Click-through rate} (CTR), for instance, is commonly used as a KPI in news 
    recommendations as it measures the fraction of recommended items that were clicked. 
    In \cite{2007GoogleNews} the authors show that personalized recommendations 
    on Google's news personalization engine led to an increase in clicks by $38\%$ compared 
    to a baseline that only recommends popular items. 

    % However, merely acessing a recommended item only grasps users' awareness towards products. In terms of appeal, 
    % \textit{Adoption and conversion} are applied measurements that 
    % indicates the depth to which users were interested in the clicked item. For instance, 
    % \textit{YouTube} uses the concept of ``long CTR'' where clicks on recommendations are only counted if the 
    % user subsequnetly watched a certain fraction of a video \cite{2010YoutubeVideoRS}. 

    % Furthermore, in web-based stores the impact of RS can be measured in terms \textit{sales and revenue}.
    % According to a statement of Amazon's CEO in 2006, about $35\%$ of their sales originated from recommendation \cite{2019BusinessValue}.
    % As we can see, the effectiveness of a RS can be measured in each step of a customers' journey on
    %  a marketing funnel as proposed by \cite{2016kotler}.

    Additionaly, using item distribution aspects to optimization goals can bring revenue to niche products, 
    highlighting one of the main competitive advantages of digital-based businesses according to the Long-Tail Theory \cite{2006LongTail}. 
    For example, in \textit{Netflix} the ``Effective Catalog Size'' is a key performance indicator that 
    expresses the amount of catalog exploration by users \cite{2016NetflixBusinessValue}.

    % Finally, user engagement and behaviour can also be measured with viewability metrics as RS changes users' experience.
    % \textit{Netflix}, for example, disclosed that ``75\%'' of what people watch is from some sort of recommendation, later 
    % revealing that the recommendation engine business value is estimated on 1 billion US dollars per year \cite{2017Beyond5stars, 2010YoutubeVideoRS}.

    Besides presenting a list of recommended items, the RS might also take a \textbf{set of actions} in order 
    to enhance human-computer interface. It is known that the design to which items as presented plays 
    a fundamental role in recommendation efficiency [21, 28], so displaying suggestions in a proper 
    frame is also part of the systems' design. 
    
    However, the RS might also have better conversational 
    features towards users other than receiving ratings and presenting items. Examples of such 
    interactions are: wishlists, giving explanations 
    about recommendations and showing complementary proposals. This puts the user more into control and 
    allow for new types of interactions \cite{2016BeyondMatrixCompletion}.  

    Finally, the \textbf{optimization timeframe} over which the goal should be optimized allows to differentiate between 
    one-shot interactions and longer or event repeated time frames. Such component varies in
    different business application. For instance, in music streaming applications, recommendations might 
    need to be repeated whereas in e-commerce scenarios the anonymous users require smaller time frames 
    be considered. 



\subsection{Structure}

\subsection{Collaborative Filtering}

    Collaborative Filtering (CF) is a popular family of techniques which underlies in the assumption 
    that users with similar relevance on past items will probably agree on evaluating future items 
    the same way [REF, melhorar]. In that sense, users' relevance to items can be estimated by their
    peers' feedback.

    Given these characteristics of CF, two groups of methods are usually portrayed in the literature: 
    the memory-based CF and the model-based CF [REF, Survey on Recommendation Systems 2016 Gaudani]. 

  \subsection{Memory-based CF}

    In the memory-based method, the utility matrix is used to calculate the similarity between 
    users or items based on past ratings or interactions. In that sense, two approaches can be made:
    the \textit{user-user} CF and the \textit{item-item} CF \cite{2011ekstrand}. 
    
    The user-user CF, also known as \textit{k-nearest neighbors} CF tries to find users' peers 
    whose past evaluation behaviour is similar to the target user. By finding these peers, their 
    feedback for an item in question is weighted by their level of agreement with the target user.  
    
    A critical design decision in implementing this approach is the choice o similarity functions. 
    Several functions have been proposed in the literature, such as the Pearson correlation, the 
    Spearman rank correlation or the Cosine similarity. An overview of these functions applied to RS
    can be found in \cite{2011ekstrand}.
    
    While effective, user-user CF suffers from scalability problems when the user-item ratio grows, which
    tipically happens in e-commerce environments (thus requiring operations on larger similarity matrices).
    To overcome this issue, \textit{item-item} CF uses similarities between the evaluation patterns 
    of items. In its overall structure, it is similar to content-based recommendation, but instead of 
    relying primarily on item's metadata, it deduces items' similarities from user preference patterns \cite{2001sarwar}.

    Several advantages can be listed for the memory-based CF. Given that mostly rely on similarity 
    functions, their algorithms can be easily implemented. Furthermore, memory-based CF are online 
    learning models, which provides great advantages to the arrival of new data and, therefore, 
    to the adaptation of concept drifts to users' preferences [REF]. Finally, explanations about the 
    recommendations can be provided to users, which has shown to increase users' trust in the system 
    [REF, ZHENG 2016].

    However, data sparsity poses a great disadvantage to these methods. Since the utility matrix is 
    sparse in most applications, the efficiency in similarity calculations is compromised by the lack 
    of sufficient data [REF, AGARRWAL, Recommender SYstems: The textbook 2016]
    
    An example of item-item CF was shown in \cite{2003amazon}, where authors use a recommendation 
    algorithm to personalize the online store for each customer at \textit{Amazon.com}, demonstrating 
    the scalability advantage of this approach.

  \subsection{Model-based CF}

  Model-based CF rely on data mining and machine learning methods to find underlying patterns on 
  users' preferences based on their feedback (although users' and items' metadata can be also used to 
  improve recommendation performance [REF]). 

  [Vantagens]

  [Desvantagens]

  [Exemplos]


\subsection{Content-Based Filtering}



\section{Recommender Systems Evaluation}

Rever artigo "Beyond Matrix Completion" para mais informações